\documentclass[Serif, 10pt, brown]{beamer}
\usepackage{booktabs,xcolor}
%\usepackage[svgnames,table]{xcolor}
%\usepackage[tableposition=above]{caption}
\usepackage{pifont}
\newcommand*\CHECK{\ding{51}}
\usepackage{array}
\newcolumntype{P}[1]{>{\centering\arraybackslash}p{#1}}
%
\usepackage{setspace,mathtools,amssymb,multirow,array,amsmath,tikz}
\usepackage[normalsize]{subfigure}
\usetikzlibrary{patterns}
\usetikzlibrary{automata,positioning,decorations.pathreplacing,decorations}

\usepackage{curves}
\usepackage{wasysym}
\usepackage{epsfig,epstopdf,graphicx}

\curvewarnfalse
%
\newtheorem{proposition}{Proposition}
\theoremstyle{example}
\newtheorem{theoremh}{Theorem}
\theoremstyle{plain}
\renewcommand{\textfraction}{0.01}
\renewcommand{\floatpagefraction}{0.99}
\newcommand{\ul}{\underline}
\newcounter{units}
%
\usepackage[round]{natbib}
 \bibpunct[, ]{(}{)}{,}{a}{}{,}%
 \def\bibfont{\small}%
 \def\bibsep{\smallskipamount}%
 \def\bibhang{24pt}%
 \def\newblock{\ }%
 \def\BIBand{and}%
%
\setbeamercovered{dynamic}
% Logo
\logo{\includegraphics[width=0.5in,keepaspectratio]{iitb_logo.png}}
%
% Setup
\mode<presentation>
	{
\usetheme[right,currentsection, hideothersubsections]{UTD}
			\useoutertheme{sidebar} \useinnertheme[shadow]{rounded}
			\usecolortheme{whale} \usecolortheme{orchid}
			\usefonttheme[onlymath]{serif}
			\setbeamertemplate{footline}{\centerline{Slide \insertframenumber/\inserttotalframenumber}}
	}
%
% Title
\usebeamercolor[fg]{author in sidebar}
\title[{DBIS}]{\sc Automatic Index Creation}
\author[\ul{Authors}]{{\bf { Saksham Rathi, Kavya Gupta, Shravan S, Mayank Kumar}}\\ {\footnotesize \hspace{0cm} (22B1003) \hspace{1cm} (22B1053) \hspace{0.5cm} (22B1054) \hspace{0.5cm} (22B0933)}}
\institute[UTD]{\sc\small CS349: DataBase and Information Systems\\ Under Prof. Sudarshan and Prof. Suraj}
\date[UCI]{Indian Institute of Technology Bombay \\ Spring 2024-25}
%
%Presentation
\begin{document}
\frame{\titlepage}
%
%
%Slides

%TOC

\begin{frame}
	\transblindsvertical
	\frametitle{Contents}
	\tableofcontents[hidesubsections]
\end{frame}
\note[itemize]{
\item Here's the overall structure of my talk today.
}

\begin{frame}{Introduction to the Problem Statement}

	\begin{itemize}
		\item Indexes are crucial for efficient query execution in relational databases.
		\item However, developers sometimes forget to create indexes for frequently queried columns.
		\item This can lead to repeated full relation scans, significantly degrading performance.
		\item {\bf Goal:} Modify the application layer of PostgreSQL to detect such patterns and automatically create indexes when beneficial.
		\item Approach:
		\begin{itemize}
			\item Track full relation scans with equality predicates.
			\item Estimate the potential benefit of an index.
			\item Automatically trigger index creation if estimated benefit outweighs the cost.
		\end{itemize}
	\end{itemize}
\end{frame}

\begin{frame}{Directory Structure}
	Here is the directory structure of the submission:
	
\end{frame}
	

\end{document}


